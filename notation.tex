\documentclass{article}
\usepackage{amsmath, amssymb, amsthm, leftindex}

\newtheorem*{defn}{}

\begin{document}


\title{Actuarial Notation}
\author{bungleship}
\date{November 19, 2023}
\maketitle
\section{Introduction and motivation}
booyah
\section{The basic deterministic model}
Let $T = \{0,\cdots,N\}$, the integers from 0 to N, be the times when a cash flow can occur.
\begin{defn}
  A cash flow vector $\vec{c} \in \mathbb{F}^{N+1}$ has a net cash flow of $c_k \in \mathbb{F}$ for each $k \in T$.
$$
\vec{c} = \begin{bmatrix}
    c_0 \\
    c_1 \\
    \vdots \\
    c_N
  \end{bmatrix}
$$
\end{defn}
\begin{defn}
  Let $s,t,u \in T$. The discount function $v\colon T \times T \to \mathbb{R}^+$ satisfies
$$
v(s,t)v(t,u) = v(s,u)
$$
We define $v(t) :=v(0,t)$.
\end{defn}
\begin{defn}
  Let $k \in T \backslash \{N\}$.The rate of interest $i_k$ and rate of discount $d_k$ for $(k,k+1)$ are defined to be
$$
\begin{matrix}
i_k = v(k+1, k) - 1\\
d_k = 1 - v(k,k+1)
\end{matrix}
$$
We will assume these rates will always be nonnegative.
\end{defn}
\begin{defn}
  The value at time $n \in T$ of $\vec{c}$ with respect to $v$ is 
$$
Val_n (\vec{c};v) = \sum_{k=0}^{N} {c_k v(n,k)}
$$
\end{defn}
Fixing $n$, $Val_n\colon \mathbb{F}^{N+1} \to \mathbb{F}$ is a linear transformation and is represented by the $1 \times (N+1)$ matrix
$$
\begin{bmatrix}
v(n,0) & v(n,1) & \cdots & v(n,N)
\end{bmatrix}
$$
The present value of a cash flow stream is $\ddot{a}(\vec{c};v) := Val_0 (\vec{c};v)$. Cash flow vectors $\vec{c}$ and $\vec{e}$ are actuarially equivalent with respect to $v$ if there is an $n \in T$ such that $Val_n(\vec{c};v) = Val_n(\vec{e},v)$.
\subsection{Vector notation}
For cash flow vectors $\vec{c},\vec{d}$, let $(\vec{c},\vec{d})$ be the juxtaposition of the two vectors. We use * to denote the Hadamard product of $\vec{c}$ and $\vec{d}$. For $r \in \mathbb{F}$ and $k \in \mathbb{Z}$, let $r_k$ be the vector of $k$ entries of r. For $i \in T$, let $\bold{e}^i$ be the zero vector plus one at position $i$. Let $\bold{j}^n = (1,2,\cdots,n-1,n)$.
\begin{defn}
Given $\vec{b}$, we define $\Delta \vec{b}$ to be
$$
\Delta{\vec{b}} =
\begin{bmatrix}
b_0 \\
b_1 - b_0 \\
b_2 - b_1 \\
\vdots \\
b_n - b_{n-1} \\
-b_n
\end{bmatrix}
$$
\end{defn}
\begin{defn}
Given $\vec{b}$ and a discount function $v$, we define $\nabla \vec{b}$ at position $k$ to be
$$
\nabla b_k = b_k - v(k, k+1)b_{k+1}
$$
\end{defn}
\subsection{Balances and reserves}
Let $k \in T$. For a cash flow vector $\vec{c}$ we let
$$
\leftindex_k{\vec{c}} =
\begin{bmatrix}
c_0 \\
c_1 \\
\vdots \\
c_{k-1} \\
0 \\
\vdots \\
0
\end{bmatrix},
\leftindex^k{\vec{c}} =
\begin{bmatrix}
0 \\
\vdots \\
0 \\
c_k \\
c_{k+1} \\
\vdots \\
c_N
\end{bmatrix}
$$
\begin{defn}
The balance at time $k \in T$ with respect to $\vec{c},v$ is
$$
B_k(\vec{c};v) = Val_k(\leftindex_k{\vec{c}};v)=\sum_{j=0}^{k-1} {c_j v(k,j)}
$$
\end{defn}
\begin{defn}
The reserve at time $k \in T$ with respect to $\vec{c},v$ is
$$
\leftindex_k{V}(\vec{c};v) = -Val_k(\leftindex^k{\vec{c}};v)=-\sum_{j=k}^{N} {c_j v(k,j)}
$$
\end{defn}
\subsection{Time shifting and the splitting identity}
\subsection{Internal rates of return}
\section{The life table}
\section{Life annuities}
\section{Life insurance}
\section{Insurance and annuity reserves}
\section{Fractional durations}
\section{Continuous payments}
\section{Select mortality}
\section{Multiple-life contracts}
\section{Multiple-decrement theory}
\section{Expenses and profits}
\section{Specialized topics}
\section{Survival distributions and failure times}
\section{The stochastic approach to insurance and annuities}
\section{Simplifications under level benefit contracts}
\section{The minimum failure time}
\section{An introduction to stochastic processes}
\section{Multi-state models}
\section{Introduction to the Mathematics of Financial Markets}
\section{Compound distributions}
\section{Risk assessment}
\section{Ruin models}
\section{Credibility theory}
\end{document}
